% Nejprve uvedeme tridu dokumentu s volbami
\documentclass[ing,male,java,dept456]{diploma}						% jednostranny dokument
%\documentclass[bc,female,java,dept456,twoside]{diploma}		% oboustranny dokument
\usepackage[czech]{babel}
\usepackage[utf8]{inputenc} 

%abstrakt, datovy slovnik
% Zadame pozadovane vstupy pro generovani titulnich stran.
\Author{David Krutký}

\Title{Optimalizace datové vrsty aplikací pro PostgreSQL}

\EnglishTitle{Optimization of a Data Layer of an Application Using PostgreSQL}

\SubmissionDate{7. května 2013}

\PrintPublicationAgreement{true}

\Thanks{Rád bych na tomto místě poděkoval všem, kteří mi pomohli, protože bez nich by tato práce nevznikla.}

\CzechAbstract{Abstrakt cz}

\CzechKeywords{Keywords cz}

\EnglishAbstract{Abstrakt en}

\EnglishKeywords{Keywords en}

% Pridame pouzivane zkratky (pokud nejake pouzivame).
\AddAcronym{TL;DR}{Too long; didn't read}

% Zacatek dokumentu
\begin{document}

% Nechame vysazet titulni strany.
\MakeTitlePages

% Asi urcite budeme potrebovat obsah prace.
\tableofcontents
\cleardoublepage	% odstrankujeme, u jednostranneho dokumentu o jednu stranku, u oboustrenneho o dve

% Jsou v praci tabulky? Pokud ano vysazime jejich seznam.
% Pokud ne smazeme nasledujici makro.
\listoftables
\cleardoublepage	% odstrankujeme, u jednostranneho dokumentu o jednu stranku, u oboustrenneho o dve

% Jsou v praci obrazky? Pokud ano vysazime jejich seznam.
\listoffigures
\cleardoublepage	% odstrankujeme, u jednostranneho dokumentu o jednu stranku, u oboustrenneho o dve


% Jsou v praci vypisy programu? Pokud ano vysazime jejich seznam.
\lstlistoflistings
\cleardoublepage	% odstrankujeme, u jednostranneho dokumentu o jednu stranku, u oboustrenneho o dve

% Zacneme uvodem
\section{Úvod}
\label{sec:Intro}
Systémy řízení báze dat (SŘBD) jsou dnes nejpoužívanějšími systémy pro ukládání dat v aplikačním softwaru. Nejrozšířenějším typem databázových systému jsou databáze relační. Na problém, který představuje komunikace aplikačního softwaru s relačním databázovým systémem, se zaměřuje tato diplomová práce.
V kapitole jsou popsány a vysvětleny základní pojmy spojené s relačními databázemi a přístupu k nim.
Důvod, který mě vedl pro vytvoření nové knihovny popisuje kapitola. 

\section{Vrstvená architektura}

Použití vrstev je jedna z nejčastějších technik využívaných pro zjednodušení komplexnosti složitých systémů \cite{fowler}. Vrstvy ve vrstvené architektuře popisují logické seskupení funkcí a komponent v aplikaci. Shlukují komponenty podle typu funkcí a ulehčují tak jejich znovupoužitelnost. Vrstvy nemusí být přitom ani fyzicky oddělené, pouze vymezují a rozdělují aplikaci do logických celků. Každá z vrstev této architektury komunikuje pouze se svými sousedními vrstvami a nestará se o další části systému. Vrstva na nižší úrovni poskytuje rozhraní, které je známé vrstvě o úroveň výše. Tato architektura tak umožňuje jednoduše nahradit nebo změnit jednotlivé vrstvy aniž by to ovlivnilo ostatní vrstvy. \\
Při vytváření aplikace založené na této architektuře je tak možno postupovat ve vývoji jednotlivých vrstev souběžně. Tohoto je dosaženo pomocí předem určeného rozhraní. Mezi další výhody patří také jednodušší údržba a testování, kde lze využít i tzv. mock-up testů. \\
Nevýhodou vrstev je, že při vysokém počtu se můžou velmi negativně projevovat na výkonu celého systému. Nejvýraznějším vlivem je potřeba rozhraní pro převod dat, jelikož každá z vrstev má vlastní logiku a interpretuje data vlastním způsobem. Dalším problémem jsou změny, které ovlivňují všechny vrstvy o úroveň výše tzv. kaskádové změny \cite{dresler}. V informačních systémech to může být například přidání databázového sloupce. \\
Vrstvenou architekturu lze aplikovat nezávisle na typu aplikace \cite{msdn}. Příkladem použití této architektury je referenční model ISO/OSI\footnote{ http://standards.iso.org/ittf/PubliclyAvailableStandards/s020269\_ISO\_IEC\_7498-1\_1994(E).zip}. \\

\subsection{Třívrstvá architektura}

Jedním z nejrozšířenějších typů vrstvené architektury v aplikacích je třívrstvá architektura. Mezi vrstvy třívrstvé architektury patří vrstvy prezentační, aplikační a datová. \\
Úkolem prezentační vrstvy je zobrazovat uživateli data ve srozumitelné a přehledné formě, reagovat na jeho požadavky a propagovat provedené změny. Prezentační vrstva může mít podobu uživatelského rozhraní v aplikaci, webové stránky nebo i jednoduchého rozhraní příkazové řádky. \\
Aplikační vrstva je srdcem aplikace. V této vrstvě se vyskytuje veškerá aplikační logika, pracuje se zde s daty, které se validují, provádějí se na nich výpočty a připravují se k uložení nebo k zobrazení. Spadají zde funkce jako autentifikace, autorizace, logování a kešování. \\
Datová vrstva zajišťuje manipulaci se zdrojem dat nebo datovým úložištěm a s daty, které tyto úložiště poskytují. Abstrahuje a centralizuje přístup k tomuto úložišti dat. Důležitým úkolem datové vrstvy je také bezpečnost dat a jejich ochrana před případnými pokusy, které se snaží k těmto datům získat přístup nebo je poškodit. Datová vrstva by se také měla snažit o co nejefektivnější výkonnost a škálovatelnost, jelikož často bývá nejvíce zatěžovaným místem v aplikaci. \\
Obrázek zobrazuje instanci této architektury v podobě frameworku Django.

\subsection{Datová vrstva aplikačního softwaru}

Velice často je třeba, aby data zpracovávána aplikací byla perzistentní. Tímto pojmem se rozumí trvalé zachování dat po zastavení provozu aplikace nebo i po výpadku systému. Jako nástroje poskytující tuto funkci jsou dnes hojně rozšířené relační databáze. \\
Princip datové vrstvy při použití relačních databází zůstává stejný. Mezi její funkce patří:
\begin{itemize}
\item správa spojení,
\item řízení paralelního zpracování a transakce,
\item synchronizace konkurenčních dat v aplikaci a v databázi,
\item validace dat,
\item formátování dat,
\item dotazování a získávání dat z databáze,
\item persistence dat,
\item správa chyb a chybových hlášení,
\item zajištění bezpečnosti dat,
\item dávkování dotazů,
\item stabilita a efektivní výkon.
\end{itemize}

Datová vrstva musí také řešit problém, který představuje nekompatibilita objektových jazyků použitých pro aplikační logiku a dotazovacích jazyků využívaných pro získávání dat z relačních databází. Tento problém označujeme pod pojmem \textit{Impedance mismatch}. Do tohoto problému spadá například převod syntaxe, převod datových typů, efektivní interpretace relací a struktur a jejich mapování mezi dvěma nekompatibilními jazyky. \\
Impedance mismatch se začal více projevovat až při použití objektově orientovaných jazyků. Jazyky používané dříve měly specializované datové struktury optimalizované pro zacházení s relačními daty, kdežto objektově orientovaným jazykům tyto struktury schází \cite{dbprogrammer}. \\

Pro zjednodušení a odstínění od těchto problémů byly vytvořeny rámce a knihovny, které se starají o tyto monotonní činnosti. Rozdíl mezi knihovnou a rámcem definuje Martin Fowler pomocí \textit{Inversion of control}. Knihovna poskytuje funkce, které je možné v aplikační logice využívat. Je tedy řízena aplikační logikou. Naopak při použití rámce je třeba umístit aplikační kód na předem určené místa například pomocí dědičnosti. Tento kód je pak volán a řízen rámcem \cite{fowler-ioc}.  \\

Datová vrstva se v angličtině označuje pod názvem \textit{Data-Access Layer} (DAL). Tímto pojmem se také někdy nazývají knihovny a rámce, které zastávají funkci této vrstvy. Často se tento název plete s \textit{Database-Abstraction Layer} (DBAL). DBAL je typ DAL, který pomocí svých funkcí abstrahuje rozdíly mezi detaily a specifikacemi podporovanými SŘBD. Umožňuje tak jednodušeji změnit SŘBD bez velkých úprav aplikačního kódu. \\

Samotná datová vrstva je velmi často rozdělena na další podvrstvy. Mezi nejprimitivnější DAL patří knihovny, které označujeme za konektory nebo ovladače. Řeší pouze část problému Impedance mismatch, poskytují základní funkce a zaměřují se pouze na jeden SŘBD. Zpravidla umožňují zadat pouze syrové SQL ve formě řetězce. Tyto konektory se většinou řídí podle specifikovaných standardů. Mnoho programovacích jazyků má určeno tyto standardy proto, aby sjednotily rozhraní a práci s těmito knihovnami, což umožňuje lehčí přechod mezi SŘBD a jednodušší vytváření DBAL. Příkladem standardu pro jazyk Python je specifikace PEP 249\footnote{http://www.python.org/dev/peps/pep-0249/} .\\
Konektory pak využívají již zmíněné DBAL a složitější rámce a knihovny snažící se o větší míru abstrakce. Tyto tak tvoří další vrstvu. \\
Nad touto vrstvou je dnes velmi často postavena ještě další funkčnost mající za cíl problém Impedance mismatch ještě více ulehčit a abstrahovat. Touto funkčností je objektově relační mapování, v angličtině \textit{Object-Relational Mapping} (ORM). ORM zle považovat za nejvyšší míru abstrakce ze všech typů DAL.  \\

%Pripadne do budoucna...
%Pro zjednodušení budu tyto knihovny považovat za první vrstvu datové vrstvy. Tyto knihovny a rámce řadím do vrstvy druhé. Řadím ji do třetí vrstvy datové vrstvy...
%Obrázek zobrazuje rozdělení známých rámců a knihoven do těchto tří vrstev, které jsem definoval výše.

Příkladem DAL jsou knihovny a rámce psycopg2, MySQL Connector/NET, PDO, JDBC, Hibernate, SQLAlchemy. Z toho lze považovat psycopg2, MySQL Connector/NET za koknektory, PDO, JDBC, Hibernate, SQLAlchemy za DBAL a Hibernate a SQLAlchemy za ORM.

\section{Objektově relační mapování}

Objektově relační (O/R) mapovaní, jak už název napovídá, mapuje entity z relačního modelu na doménové objekty v aplikační logice nejčastěji pomocí objektů nebo tříd. \\
Pod pojmem doména rozumíme u aplikace oblast z reálného světa, pro kterou se tato aplikace snaží vyřešit nebo zjednodušit specifický problém. Doména vymezuje rozsah působnosti aplikace a určuje doménové objekty. Tyto objekty se v objektově orientovaném jazyce modelují nejčastěji pomocí tříd. Například, pokud je doména nákup a prodej zboží, za doménové objekty lze považovat kupujícího a objednávku. Martin Fowler definuje ve své publikaci několik návrhových vzorů, které popisují modelování doménových objektů a přístup k doménové logice. Mezi tyto vzory patří \textit{Transaction Script}, \textit{Domain Model} a \textit{Table Module} \cite{fowler}. Popis těchto vzorů je nad rámec této diplomové práce. Tyto návrhové vzory se však berou v potaz při návrhu objektově relačního mapování. \\
Podobně jako u DAL se i pod zkratkou ORM někdy označují rámce implementující O/R mapovaní. Tyto rámce kromě O/R mapování obsahují funkce potřebné pro komunikaci s relační databází popsané výše (v sekci). Někdy i využívají existujících DAL, které tyto funkce obsahují a přidávají k nim O/R mapování. Většinou obsahují také DBAL. \\

\subsection{Rozhraní datové vrstvy}

Každé ORM definuje způsob, kterým datová vrstva komunikuje s aplikační vrstvou. Tímto rozhraním ORM také abstrahuje přístup k databázi a z části určuje i architekturu doménové logiky. Martin Fowler pojmenoval několik základních typů a návrhových vzorů pro toto rozhraní jako \textit{Table Data Gateway}, \textit{Row Data Gateway}, \textit{Active Record} a \textit{Data Mapper} \cite{fowler}. Ve všech případech jsou v objektově orientovaném jazyce jako rozhraní využity třídy, pomoci kterých ORM abstrahuje tabulky, pohledy, dynamické 	dotazy a dotazy zapouzdřené v uložených procedurách. Vzory se mezi sebou liší způsobem jakým jsou tyto třídy využity a typem abstrakce.

\subsubsection{Table Data Gateway}

Table Data Gateway využívá třídy jako brány k tabulkám. Tabulka nebo podobný databázový objekt je namapována pomocí třídy. Třída obsahuje metody pouze pro vyhledávání v datech, ukládání a mazání dat. Jedna instance této třídy přitom poskytuje rozhraní ke všem záznamům v tabulce.  Ke konkrétnímu řádku tabulky se lze dostat jen předáním klíče, jedinečného identifikátoru. Tento návrhový vzor je ve většině případů bezstavový. \\
Každá metoda třídy skrývá za svým rozhraním SQL. Nejčastěji komunikuje přímo s databázovým ovladačem a předává mu vytvořené SQL. Data jsou pak vrácena ve formě v jaké je vrací ovladač, v objektu typu Record Set. Record Set je generická reprezentace relačních dat v paměti, většinou podobná poli nebo seznamu. Metody pro vyhledávání v datech pak vracejí vždy tento objekt i pokud se jedná pouze o jediný záznam. Vrácená data navíc neobsahují žádnou referenci k tomuto rozhraní. Toto může být problém a proto se vrácená data někdy obalují dalším objektem. Tento objekt se označuje jako Data Transfer Object \cite{fowler}. \\
Výhodou při použití tohoto vzoru je, že jeho rozhraní lze využít jak pro manipulaci relačních dat, tak pro uložené procedury. Také shlukuje SQL na jedno místo, což zjednodušuje optimalizaci. Table Data Gateway lze lehce využít jako rozhraní datové vrstvy při použití vzorů Table Module nebo Transaction Stript v aplikační logice. \\
Příkladem tohoto návrhového vzoru je obrázek \ref{fig:TableDataGateway}. Na obrázku je třída PersonGateway, která slouží jako brána k databázové tabulce Person.

\InsertFigure{tabledatagateway.png}{80mm}{Třída v návrhovém vzoru Table Data Gateway \cite[str. 148]{fowler}}{fig:TableDataGateway}

\subsubsection{Row Data Gateway}

Row Data Gateway abstrahuje tabulku pomocí třídy, její záznamy pomocí instancí této třídy a její sloupce pomocí atributů. Třídy neobsahují žádnou aplikační logiku a slouží pouze jako brána k datům. Jedna instance odpovídá jednomu záznamu. Lze se tak jednoduše dostat ke konkrétnímu řádku. Instance má navíc povědomí o celém rozhraní, není tedy třeba používat Data Transfer Object. \\
Instance obsahuje metody pro manipulaci s daty řádku, jejich mazání a ukládání. Datové typy atributů třídy jsou implicitně převáděny na datové typy konkrétních sloupců tabulky. Na základě tříd a jejich atributů, lze i jednoduše pomocí nástrojů ORM vytvořit relační schéma v databázi. \\
Ke každé třídě mapující tabulku je vytvořena další třída, která slouží k vyhledávání záznamů pro konkretní tabulku. Tato třída se v pozadí chová podobně jako třída v návrhovém vzoru Table Data Gateway, je tak možné využívat i některých výhod tohoto vzoru. Namísto objektu Record Set však třída pro vyhledávání vrací instanci třídy mapující konkrétní tabulku. Pro získání záznamu je tedy nejdříve nutné využít vyhledávací třídu. \\
Instanci návrhového vzoru Row Data Gateway zobrazuje obrázek \ref{fig:RowDataGateway}. Na obrázku je třída PersonFinder, která vyhledává data v databázové tabulce Person a vrací instanci třídy PersonGateway. PersonGateway pak slouží jako rozhraní pro konkrétní řádek tabulky Person. \\

\InsertFigure{rowdatagateway.png}{40mm}{Ukázka použití návrhového vzoru Row Data Gateway \cite[str. 154]{fowler}}{fig:RowDataGateway}

\subsubsection{Active Recod}

Active Record je návrhový vzor podobný Row Data Gateway. Stejně jako Row Data Gateway mapuje struktury v datové vrstvě velmi úzce se strukturou databáze. Třídy tak korespondují s tabulkami v databázi a jejich instance a atributy s řádky a sloupci tabulek. Oproti Row Data Gateway však třídy obsahují navíc i aplikační logiku a většinou jsou tak považovány i za doménové objekty. \\
\uv{Třída ve vzoru Active Record typicky obsahuje metody, které mají za úkol následující:
\begin{itemize}
\item Vytvořit instanci třídy z dat řádku vráceného na SQL dotaz
\item Vytvořit instanci třídy, která bude později použita pro vložení nového záznamu do tabulky
\item Statické metody, které obalují často používané SQL dotazy pro vyhledávání dat a které vracejí instance třídy
\item Aktualizovat databázi daty instance
\item Metody pro získání a nastavování hodnot atributů
\item Implementace části aplikační logiky
\end{itemize}}\cite[str. 160]{fowler} \\
Tento návrhový vzor je intuitivním přístupem, pokud při návrhu databázového schématu korespondují tabulky s doménovými objekty. Proto a pro svoji jednoduchost je dnes velmi rozšířeným a často používaným vzorem při návrhu ORM. Používá se hlavně pro jednoduché systémy, kterým umožňuje rychlou implementaci rozhraní datové vrstvy. \\
Mezi nevýhody návrhového vzoru Active Record patří mapování vztahů a dědičnosti, které z části rozbíjí jeho principy a přispívají tak ke špatné struktuře kódu a chybám, které z toho plynou. Podobný problém nastává při složitější aplikační logice, jelikož pak třídy výrazně nabudou na objemu a zastávají nemalý počet funkcí. Toto vede ke špatně testovatelným objektům a nevyhnutelně k chybám. Active Record se nevyplatí používat, pokud se doménové objekty liší od struktur v databázi nebo pokud by v budoucnu mohlo k takovýmto změnám dojít.

\subsubsection{Data Mapper}

Data Mapper se využívá při složitější aplikační logice, při rozdílném objektovém a relačním schématu nebo často i pokud je třeba pracovat s již existující databází. Doménové objekty v aplikační logice se při použití tohoto vzoru nemusí shodovat s databázovým schématem. Lze je vytvořit naprosto nezávisle. Je tak možné kdykoliv změnit databázové schéma nebo struktury v aplikační logice z čehož vyplývá mnoho výhod. \\
Nevýhodou tohoto přístupu je, že přidává další vrstvu mezi databázi a aplikační vrstvu. Toto se může negativně projevovat na celkovém výkonu a složitosti systému, proto se Data Mapper nedoporučuje používat pro jednoduché případy, které lze vyřešit například pomocí vzoru Active Record.\\
Data Mapper komunikuje s aplikační vrstvou pomocí mapovací třídy, která poskytuje rozhraní k získávání doménových objektů. Mapovací třída může abstrahovat jednu databázovou tabulku, její část nebo i více tabulek. Většinou je výhodnější pro jednu doménovou třídu vytvořit vždy jednu mapovací třídu. \\
Obrázek \ref{fig:DataMapper} zobrazuje jednoduchý příklad návrhového vzoru Data Mapper. Na obrázku je  mapovací třída PersonMapper, která získává data z databáze a vrací je ve formě instancí doménové třídy Person. Implementace získávání dat je před třídou Person skrytá a databázové schéma může vypadat jakkoliv. \\
Data Mapper využívá pro spravování doménových objektů strukturu pojmenovanou Identity map \cite{fowler}. Tato struktura se stará synchronizaci doménových objektů a dat v paměti. Předchází prováděním zbytečných dotazů do databáze, pokud daný objekt již v paměti existuje a zajišťuje, aby v takovém případě místo nového objektu byla vrácena reference na existující objekt. \\
Tento návrhový vzor má smysl používat jen tehdy, pokud se aplikační logika řídí vzorem Domain Model. Při použití jiných vzorů přináší Data Mapper mnoho nadbytečných komplikací a jeho použití se nevyplácí. \\

\InsertFigure{datamapper.png}{105mm}{Ukázka použití návrhového vzoru Data Mapper \cite[str. 164]{fowler}}{fig:DataMapper}

\subsection{Mapování vztahů}

V relační databázi se používají cizí klíče pro vyjádření relací, ale objekty v objektově orientovaných jazycích dosahují vazeb za pomocí referencí na konkrétní objekt. ORM tento problém řeší mapováním vztahů.\\
Mapování vztahů a relací je funkce, která dle mého názoru definuje ORM. ORM řídící se podle návrhových vzorů Row Data Gateway, Active Record a Data Mapper potřebují tuto funkci, jelikož používají instance tříd, které abstrahují databázové tabulky, jako záznamy tabulky. ORM se poskytuje tři druhy mapování relací.
Knihy, které v databázi obsahují cizí klíč na autora jsou pomocí ORM namapovány na objekt autora, který obsahuje kolekci s referencemi na jednotlivé objekty knih. \\
Problém nastává při získávání objektu autora z databáze. ORM v souvislosti s mapováním vztahů využívají \textit{Lazy Loading}. Lazy Loading je funkce, která zajišťuje, že při načtení objektu z databáze do paměti se data, které tvoří relaci, nenahrají ihned, ale až při přístupu k dané relaci. Při absenci Lazy Loading, hrozí, že by se do paměti načetlo velké množství dat, v nejhorším případě celá databáze.

\subsection{Dědičnost}

\section{Proc novy framework}

Důvody, které mě vedly k vytvoření knihovny
%- vlastní i cizí zkušenosti s pomalostí ORM
%- zaměřeno na více databází
% je nutne znat SQL

\section{Srovnani frameworku a jejich funkce pro zrychleni vykonu}

\section{DAL Framework}

\section{Implementovane funkce}

%-vysvetleni syntaxe, vsech mechanismu do podrobna
%-komentare ve zdrojovem kodu a principy clean code

\section{Dalsi funkce, ktere budou v budoucnu implementovany}

\section{Porovnani rychlosti DAL s ORM a jinyma}

\section{Závěr}
\label{sec:Conclusion}


\begin{thebibliography}{99}

\bibitem{fowler} Martin Fowler Patterns of en. application
\bibitem{msdn} http://msdn.microsoft.com/en-us/library/ee658109.aspx
\bibitem{dipl.net} Diplomova prace orm pro .net
\bibitem{fowler-ioc} http://martinfowler.com/bliki/InversionOfControl.html
\bibitem{dbprogrammer} http://database-programmer.blogspot.cz/2010/12/historical-perspective-of-orm-and.html
\bibitem{dresler} http://www.robertdresler.cz/2011/04/vicevrstve-architektury-aplikaci.html

\end{thebibliography}


\appendix
\section{Grafy a měření}
Tohle je příloha k práci. Většinou se sem dávají grafy, tabulky, které by vzhledem
ke svému počtu překážely v textu diplomky.
\clearpage


\end{document}
